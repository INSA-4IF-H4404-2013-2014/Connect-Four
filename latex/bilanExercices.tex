\chapter{Bilan des exercices}

\section{Prédicats}

La particularité de Prolog réside dans le fait qu'il n'y a plus d'itérations comme
un langage de programmation conventionnel. Tout n'est que prédicat. Ainsi, la
méthode de programmation est très différente. Fini le traitement de données
dans des variables que l'on met à jour, car la programmation par prédicats permet
simplement de lier des propriétés entre des variables et/ou constantes.

Considérons pour la suite, le prédicat~:
\[(membre(X, L) \Leftrightarrow vrai) \Leftrightarrow X \in L\]


\section{Vérification de propriétés}

La vérification de propriétés permet de s'assurer qu'une (ou plusieurs) constantes
vérifient un ensemble de prédicats. Considérons le code ci-dessus.

Alors on a à l'exécution~:

\begin{lstlisting}[language=Prolog,frame=single]
?- membre(1, [1, 2, 3]).
true

?- membre(4, [1, 2, 3]).
false
\end{lstlisting}

En effet, à la première interrogation, on vérifie le prédicat $1 \in [1, 2, 3]$,
ce qui est vrai, d'où la réponse de Prolog «~\textit{true}~». La propriété est alors vérifiée, 
renvoyant ainsi vrai. Tandis que la seconde interogation $4 \in [1, 2, 3]$ est fausse car 
$4 \notin [1, 2, 3]$, d'où la réponse «~\textit{false}~».


\section{Recherche de solutions}

La recherche de solution consiste à définir des propriétés entre des
objects et/ou constantes. Par exemple~:
\[
	X \in [1, 2, 3]
\]

 Ce qui en Prolog donne :
\begin{lstlisting}[language=Prolog,frame=single]
?- membre(X, [1, 2, 3]).
\end{lstlisting}

Ainsi a l'éxécution, Prolog est capable d'évaluer les solutions de $X$ grace a
cette propriétée ainsi définie :

\begin{lstlisting}[language=Prolog,frame=single]
?- membre(X, [1, 2, 3]).
X = 1;
X = 2;
X = 3;
false
\end{lstlisting}


\section{Recherche de solutions d'un système}

Une propriétée sur une variable par exemple, peut être définit par plusieur
prédicats. Par exemple~:
\[
    \left\{  
    \begin{array}{c}
        L \in [1, 2, 3]\\
        L \in [3, 4, 2]\\
    \end{array}
    \right .
\]

Cela revient simplement a l'écriture en Prolog :
\begin{lstlisting}[language=Prolog,frame=single]
?- membre(X, [1, 2, 3]), membre(X, [3, 4, 2]).
X = 2;
X = 3;
false
\end{lstlisting}


\section{Recherche de solutions non-déterministe}

La dangeureusitée de de la recherche de solution, est qu'il est possible qu'une
infinitée de solutions vérifient une meme propriétée. Considerons par exemple
le code suivant~:

\begin{lstlisting}[language=Prolog,frame=single]
?- membre(1, L).
\end{lstlisting}

Cette est équivalent à $1 \in L$. Mais alors, combien de listes
pourraient vérifié cette propriété ?

\textbf{Initialisation}~: Une liste telle que $[1, 2]$ vérifie cette propriété.

\textbf{Hérédité}~: En notant $cat(A, B)$ la concatenation de deux listes $A$ et
$B$,\\
Soit une liste $L$ telle que $1 \in L$,\\
Alors $\forall X \in \mathbb{N} / 1 \in cat([X], L)$

\textbf{Conclusion}~: Il éxiste une infinité de solutions et Prolog va essayer
de toutes les générer, causant une exception du au manque de mémoire de la
machine.

\begin{lstlisting}[language=Prolog,frame=single]
?- membre(1, L).
L = [1|_G2214] ;
L = [_G2213, 1|_G2217] ;
L = [_G2213, _G2216, 1|_G2220] ;
L = [_G2213, _G2216, _G2219, 1|_G2223] ;
L = [_G2213, _G2216, _G2219, _G2222, 1|_G2226] ;
L = [_G2213, _G2216, _G2219, _G2222, _G2225, 1|_G2229] ;
...
\end{lstlisting}


\section{Programation de prédicats triviaux}

Au paravant, nous fesions que utiliser des prédicats, mais bien entendu,
l'objectif est de pouvoir coder les siens. C'est ici que la méthode de programmation diffère
complement de la programmation itérative. Nous devons proceder avec une démarche
d'analyse mathématique. Pour cela, interessons nous à la ré-écriture de
\[(membre(X, L) \Leftrightarrow vrai) \Leftrightarrow X \in L\]

Tout d'abort, on remarque que $X \in [X, ...]$, ou autrement dit~:
$X \in (L = cat([X], L1))$ avec $L1$ une autre liste. Alors on en deduit le
premier prédicat :
\begin{lstlisting}[language=Prolog,frame=single]
membre(X, [X|_]).
\end{lstlisting}

On remarque par ailleur que~:
\[\forall (X, Y) \in \mathbb{E}^2, X \in L \Rightarrow X \in cat([Y], L)\]
Ainsi on en déduit le second prédicat~:
\begin{lstlisting}[language=Prolog,frame=single]
membre(X, [_|L]) :- membre(X, L).
\end{lstlisting}


\section{Prédicats avec calculs arithmétiques}

On se propose d'implementer un predicat permetant de verifier a un element a une
position donnée dans une liste : $X = L[I]$

Exemple :
\begin{lstlisting}[language=Prolog,frame=single]
?- elementAtPos(2, hello, [hi, hello, bye]).
true

?- elementAtPos(3, hi, [hi, hello, bye]).
false
\end{lstlisting}


Tout d'abort, le premier cas evidents survient lorsque la list $L$ est vide, le
predicat doit etre faut~:
\begin{lstlisting}[language=Prolog,frame=single]
elementAtPos(_, _, []) :- fail.
\end{lstlisting}
Notez que cette ligne est facultative, car definir un predicat comme echouant
(fail) revien a ne pas le definir. Nous le metons ici simplement pour
illustrer la demarche.

Vien ensuite le cas evident ou nous voulons tester le premier element de la
liste. Ainsi :
\begin{lstlisting}[language=Prolog,frame=single]
elementAtPos(1, X, [X|_]).
\end{lstlisting}

Enfin, il demeure le cas ou ce n'est pas le premier element. Alors, l'idee, est
de recursivement prouver $elementAtPos/3$ mais en retirant le premier element
a chaque foie~:
\begin{lstlisting}[language=Prolog,frame=single]
elementAtPos(I, X, [_|L]) :-
    I > 1, I1 is I-1, elementAtPos(I1, X, L).
\end{lstlisting}
