
\chapter{Bilan des exercices}

\section{Propriétés : variables liés par des predicats}

La puissance de prolog vien du principe de reprouvabilite. Et les predicats ne
representes alors que des proprieter sur des variables que l'on leurs passe en
parametre. Considerons par exemple le code ci dessous :

\begin{lstlisting}
membre(X, [X|_]).
membre(X, [_|L]) :- membre(X, L).
\end{lstlisting}

Alors on a à l'éxécution :

\begin{lstlisting}
?- membre(1, [1, 2, 3]).
true

?- membre(4, [1, 2, 3]).
false
\end{lstlisting}

En effet, a la premiere intérogation, on vérifie le prédicat 1 apartenant à
[1, 2, 3]. La proprieté etre ces deux paramêtres est alors vérifié, renvoyant
ainsi vrai.


\section{Reprouvabilitée}

La reprouvabilitees consiste maintenant de definir des propriete entre des
variables/constantes. Par exemple :

\begin{lstlisting}
membre(L, [1, 2, 3]).
\end{lstlisting}

On lis a ce moment que L compose la liste constante [1, 2, 3]. Ainsi a
l'execution, Prolog peut alors evaluer les solutions de L grace a la propriete
ainsi defini ci dessus.

\begin{lstlisting}
?- membre(L, [1, 2, 3]).
L = 1;
L = 2;
L = 3;
false
\end{lstlisting}


\section{Reprouvabilitée non-déterministe}

La dangeureusitee de de la reprouvabilitée, est qu'il est possible qu'une
infinitée de solutions vérifient une meme propriétée. Considerons par exemple
le code suivant :

\begin{lstlisting}
?- membre(1, L).
\end{lstlisting}

Cette ecriture dit alors que 1 compose (la liste) L ce qui est equivalent a dire
que la list L est composée de un (ou plusieurs) 1. Mais alors, combien de listes
pourraient vérifié cette propriété ? Une infinité bien sur. Et pour preuve,
voici l'execution de Prolog :

\begin{lstlisting}
?- membre(1, L).
L = [1|_G2214] ;
L = [_G2213, 1|_G2217] ;
L = [_G2213, _G2216, 1|_G2220] ;
L = [_G2213, _G2216, _G2219, 1|_G2223] ;
L = [_G2213, _G2216, _G2219, _G2222, 1|_G2226] ;
L = [_G2213, _G2216, _G2219, _G2222, _G2225, 1|_G2229] ;
...
\end{lstlisting}
