\chapter{Projet~: Puissance~4}

\section{Règles du jeux}

Le puissance 4 est un jeu de societé à deux joueurs. Chaque joueur doit,
chacun son tour, insérer un jeton de sa couleur dans une des sept
colonnes côte à côte, chacune ayant une capacité maximale de
six jetons. Ce pion doit, bien entendu, tomber jusqu'à la case la plus basse inocupée de sa colonne. Les cas d'un jeton ne tombant pas jusqu'en bas ou dans la colonne choisie ne comptent pas comme coups joués. Le but du jeu est d'aligner verticalement, horizontalement 
ou en diagonale 4 jetons de sa couleur avant l'adversaire.


\section{But du joueur idéal}

Dans le cas d'un joueur idéal, le but est simplement, après avoir aligné 3 jetons,
de prévoir de jouer le 4\up{ème} au tour suivant. Mais, comme l'adversaire pourrait casser la ligne en jouant à cet endroit
lors de son tour. L'objectif du joueur idéal est donc de réaliser
au moins deux alignements de 3 jetons en un coup. Laissant ainsi le joueur adverse
contre l'inévitable fatalité~: Il ne pourra plus contrer ces tentatives en un seul jeton.


\section{Travail realisé}

En plus de l'implémentation du module de méchanisme de jeux et réalisation des
tests unitaires, nous avons implementé 4 intelligences artificielles ayant des stratégies différentes~:

\begin{itemize}

    \item joueur aléatoire~;
    \item joueur aléatoire muni d'heuristiques~;
    \item joueur parcourant l'arbre des possibilités~;
    \item inteligence artificielle apprenant par \textbf{moteur d'inférence}, de ses
    échecs precédents.

\end{itemize}

Mais également~:

\begin{itemize}

    \item interface utilisateur en ligne de commande pour jouer une partie~;
    \item module de tournois générant des statistiques~;
    \item module d'entrainement du moteur d'inférence~;
    \item module d'étude de l'apprentissage du moteur d'inférence~;
    \item sauvegarde et chargement de la base de connaissances du moteur d'inférence.

\end{itemize}
