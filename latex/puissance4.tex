\chapter{Projet~: Puissance~4}

\section{Règles du jeux}

Le puissance 4 est un jeux de societé à deux joueurs. Chaque joueurs doivent,
chacun leurs tour, inserer un jeton de leurs couleur dans une des sept
colones cote à cote, chacunes aillant une capacité maximal de
six jetons. Le but du jeu est d'aligner verticalement, horizontalement ou
ou en diagonal, 4 jetons de sa couleur avant l'adversaire.


\section{But du joueur idéal}

Dans le cas d'un joueur idéal, le but n'est simplement d'aligner 3 jetons precedament,
puis prévoir de jouer le 4\up{ème} au tour suivant. En effet, l'adversaire pourrait bloquer cet alignement,
lorsque c'est a son tour de jouer. L'objectif du joueur idéal est donc de réaliser
au moins deux alignements de 3 jetons en un coup. Laissant ainsi le joueur adverse
contre l'inévitable fatalité~: Il ne peut plus contrer ces alignements en un seul jeton.


\section{Travail realisé}

En plus de l'implémentation du module de méchanisme de jeux et réalisation des
tests unitaire, nous avons implementé 4 joueurs aillant des stratégies différentes~:

\begin{itemize}

    \item joueur aléatoire~;
    \item joueur aléatoire munis d'heuristiques~;
    \item joueur parcourant l'arbre des possibilités~;
    \item inteligence artificielle apprenant par \textbf{moteur d'inférence}, de ses
    échecs precédent.

\end{itemize}

Mais aussi~:

\begin{itemize}

    \item interface utilisateur en ligne de commende pour jouer une partie~;
    \item module de tournoit générant des statistiques~;
    \item module d'entrainement du moteur d'inférence~;
    \item module d'étude de l'apprantissage du moteur d'inférence~;
    \item sauvgarde et chargement de la base de connaissances du moteur d'inférence.

\end{itemize}
